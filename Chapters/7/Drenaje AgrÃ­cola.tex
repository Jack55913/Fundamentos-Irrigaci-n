\chapter{Drenaje Agrícola}

Cultivo de Maíz, con un suelo Arcilloso de las características siguientes:
\begin{itemize}
    \item CC= 35\%
    \item PMP= 17\%
    \item Da= $1.25 g/cm^3$
    \item Pr= 60cm
\end{itemize}

Cultivo de Maíz, con un suelo Franco de las características siguientes:
\begin{itemize}
    \item CC= 22\%
    \item PMP= 10\%
    \item $Da= 1.4 g/cm^3$
    \item Pr= 60 cm
\end{itemize}

Cultivo de Maíz, con un suelo Arenoso de las características siguientes:
\begin{itemize}
    \item CC= 9\%
    \item PMP= 4\%
    \item $Da= 1.65 g/cm^3$
    \item Pr= 60 cm
\end{itemize}

La eficiencia de aplicación en estos tres tipos de suelo, es de $EA=45\%$
\begin{equation}
    \Delta H= \frac{R}{\mu}
\end{equation}

Calculando la porosidad:
\begin{equation}
    P_t = \frac{Dr - Da}{Dr} \cdot 100
\end{equation}
La densidad real varía entre 2.6 y $2.7 g/cm^3$
\begin{align*}
    &Lr = \frac{CC - PMP}{100} \cdot Da \cdot Pr = 13.5\\
    &L_b =\frac{LN}{EA} = \frac{13.5}{0.45}= 30cm\\
    &R = 30cm - 15cm = 16.5cm\\
    &\Delta h = \frac{16.5}{0.18} = 19.6cm\\
    &L_b = \frac{13.5}{0.95}= 14.2
\end{align*}
% Calculando 49.28

La planta deja de absorver K, N, $P_2O_5$, Ca, Mg

\begin{itemize}
    \item Gris azul: SFe
    \item Sulfuro de hidrógeno: $SH_2$ gas tóxico
    \item Manchas amarillas: Buena aireación
    \item Metano: $CH_4$
\end{itemize}
\begin{equation}
    NO_3\implies NO_2\implies N_2O\implies NH_3\implies N_3
\end{equation}
% El azotobacter














