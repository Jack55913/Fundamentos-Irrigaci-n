\chapter{Operación de Canales}
\section{Conceptos básicos de operación y planeación de canales}
\subsection{Eficiencias del agua de riego}
\begin{definition}[Eficiencia en el uso del agua]
    Es la relación entre el volumen de agua utilizado con un fin determinado y el volumen extraído o derivado de una fuente de abastecimiento con ese mismo fin
    \begin{equation}
        Ef = \frac{V_u}{V_e}
    \end{equation}
    \begin{notation}
        \begin{itemize}
            \item $Ef$= Eficiencia adimensional
            \item $_Vu$= Volumen extraído $L^3$
            \item $_Ve$= Volumen extraído de la fuente de abastecimiento $L^3$
        \end{itemize}
    \end{notation}
\end{definition}
Componentes de la eficiencia de uso para riego (Comisión internacional de Riego y Drenaje)
\begin{itemize}
    \item Almacenamiento
    \item Conducción
    \item De uso de riego en la parcela
\end{itemize}
\begin{definition}[Eficiencia de Almacenamiento]
    Es la relación entre el volumen que se deriva para riego ($d_v$), entre el volumen que entra a un vaso de almacenamiento ($V_e$) para el mismo fin
    \begin{equation}
        Es = \frac{V_d}{V_e}
    \end{equation}
\end{definition}
\begin{definition}[Eficiencia de Conducción]
    Es la relación entre el volumen de agua que se entrega a las parcelas para riego ($V_p$) y el volumen que se deriva de la fuente de abastecimiento ($V_d$)
    \begin{equation}
        E_c = \frac{V_p}{V_d}
    \end{equation}
    \begin{notation}
        \begin{itemize}
            \item $V_u$= Volumen de agua útil almacenado en la zona de exploración de las raíces de las plantas
            \item $V_p$= Volumen recibido en la parcela
        \end{itemize}
    \end{notation}
\end{definition}
\begin{align*}
    E_i = \frac{V_u}{V_e} = \frac{V_d}{V_e} \times \frac{V_p}{V_d} \times \frac{V_u}{V_p}\\
    E_i = E_s \times E_c \times E_a
\end{align*}
\begin{example}
    Eficiencias del Distrito de Riego del Río Mayo, Sonora

    \begin{itemize}
        \item 34 años de operación de la presa ``Adolfo Ruiz Cortínez''
        \item Entrada promedio de $980hm^3$ y salidas de $860 hm^3$
        \item Eficiencia de almacenamiento:\footnote{Recordando que $1hm^3=1,000,000m^3=1,000,000,000$ litros} \begin{equation*}
            E_s = \frac{V_d}{V_e} \cdot 100 = \frac{860}{980} \cdot 100 = 88\%
        \end{equation*}
        De los $120 hm^3$ que en promedio anual se han perdido: \begin{itemize}
            \item $80 hm^3$ lo fueron por derrames por el vertedor
            \item $40hm^3$ por evaporación en el vaso
            \item Las pérdidas por filtración no son significativas
        \end{itemize}
    \end{itemize}
\end{example}
Las pérdidas por conducción son debidas a:
\begin{itemize}
    \item Infiltración
    \item Evaporación
    \item Fugas
    \item Por manejo del agua en la red de distribución (operación)
\end{itemize}
\begin{align*}
    &E_c = \frac{V_p}{V_d} \cdot 100 = \frac{545hm^3}{860hm^3} \cdot 100 = 63\%\\
    &E_c = \frac{Q_p}{Q_d} \cdot 100 = \frac{20.3 \frac{m^3}{s}}{32 \frac{m^3}{s}} \cdot 100 = 63.4\%
\end{align*}
Para el año analizado, el volumen total extraído de la presa fue de $860hm^3$ ($V_d$), el servido $V_p=545hm^3$, las pérdidas totales de conducción de $315hm^3$, divididas en $180hm^3$ de intrínsecas y $135hm^3$ por operación
\begin{table}[h!]
    \begin{tabular}{@{}ccc@{}}
    \toprule
    \multirow{2}{*}{Tipo de pérdidas} & \multicolumn{2}{c}{Pérdidas} \\
                                      & $hm^3$         & \%          \\ \midrule
    Evaporación                       & 42.7           & 13.6        \\
    Infiltración                      & 102.4          & 32.5        \\
    Fugas                             & 33.4           & 10.6        \\
    Manejo                            & 136.5          & 43.3        \\
    Total                             & 315            & 100         \\ \bottomrule
    \end{tabular}
    \caption{Análisis de la presa Adolfo Ruíz Cortínez}
    \label{taboc1}
\end{table}
Eficiencia de aplicación del riego ($E_a$)
\begin{equation*}
    E_a = \frac{V_u}{V_p} = \frac{409hm^3}{545hm^3} \cdot 100 = 75\%
\end{equation*}
La eficiencia total de uso de agua para la irrigación ($E_i$)
\begin{align*}
    &E_i = E_s \cdot E_c \cdot E_a = 0.88 \cdot 0.634 \cdot 0.75 = 0.42 = 42\%\\
    &E_i = E_c \cdot E_a = 0.634 \cdot 0.75 = 0.476 = 47.6\%
\end{align*}